\documentclass[a4paper,11pt]{article}
% 中文支持
\usepackage[UTF8]{ctex}
\usepackage{geometry}
\geometry{left=2.5cm,right=2.5cm,top=2.5cm,bottom=2.5cm}
% 基本包
\usepackage{amsmath}
\usepackage{graphicx}
\usepackage{booktabs}
\usepackage{fancyhdr}
\usepackage{xcolor}
\usepackage{listings}
\usepackage{hyperref}

% 代码高亮配置
\lstset{
    basicstyle=\ttfamily\small,
    backgroundcolor=\color{gray!10},
    frame=single,
    numbers=left,
    numberstyle=\tiny\color{gray},
    keywordstyle=\color{blue},
    commentstyle=\color{green!60!black},
    stringstyle=\color{red},
    breaklines=true,
    breakatwhitespace=true,
    tabsize=4
}

% 页眉设置
\pagestyle{fancy}
\fancyhf{}
\fancyhead[L]{\textbf{\color{green!30!black}{中山大学人工智能学院}}}
\fancyhead[R]{\thepage}
\renewcommand{\headrulewidth}{0.4pt}

\begin{document}
\begin{center}
    {\LARGE \textbf{人工智能实验原理作业报告}} \\[0.3cm]
    {\Large \textbf{实验题目:} [实验题目] } \\[0.2cm]
    \textbf{姓名:} [姓名] \quad \textbf{学号:} [学号] \quad \textbf{班级:} [班级] \quad \textbf{日期:} \today
\end{center}

% 渐变式分隔
\begin{center}
\makebox[\textwidth]{\dotfill}
\end{center}

{\noindent \Large \textbf{\textcolor{red}{模板标题仅供参考,请根据具体实验要求自行修改}}}

\section{实验目的}

从实践中学习收集数据、处理数据集的方法;学习LoRA微调的原理,动手实践LoRA微调。

\section{实验原理}
LoRA微调的原理

% lora微调公式
\begin{equation}
    f(x) = \sum_{i=1}^{n} w_i x_i + b
\end{equation}

\section{实验环境}
\begin{itemize}
    \item \textbf{操作系统:} Windows 11 
    \item \textbf{编程语言:} Python 3.10.16
    \item \textbf{主要库:} TensorFlow, PyTorch
    \item \textbf{开发环境:} VScode
\end{itemize}

\section{实验内容与步骤}
\subsection{数据预处理}
来自开源数据集。URL: https://github.com/thu-coai/KdConv.git

\subsection{模型构建}
模型使用26M的MiniMind2-small的基座模型和训练出的LoRA权重

LoRA微调的参数:
epoch = 10
batch_size = 32
learning_rate = 0.0001
accumulation_steps = 1
grad_clip = 1
warmup_iters = 0
hidden_size = 512
num_hidden_layers = 8
max_seq_len = 512
use_moe = False


% 代码块示例
\begin{lstlisting}[language=Python, caption=示例代码]
import numpy as np
from sklearn.model_selection import train_test_split

# 数据加载和预处理
X_train, X_test, y_train, y_test = train_test_split(X, y, test_size=0.2)
\end{lstlisting}

\subsection{训练过程}
(说明训练参数、优化器选择和训练过程...)
LoRA微调的参数:
epoch = 10
batch_size = 32
learning_rate = 0.0001
accumulation_steps = 1
grad_clip = 1
warmup_iters = 0
hidden_size = 512
num_hidden_layers = 8
max_seq_len = 512
use_moe = False

(优化器选择???)

训练过程:待从 记录.md 插入

\section{实验结果}
\subsection{性能指标}
展示实验结果,包含关键性能指标:

\begin{table}[h]
    \centering
    \caption{模型性能对比}
    \begin{tabular}{lccc}
        \toprule
        模型 & 准确率 & 精确率 & 召回率 \\
        \midrule
        模型A & 0.85 & 0.82 & 0.88 \\
        模型B & 0.87 & 0.84 & 0.90 \\
        模型C & 0.89 & 0.86 & 0.92 \\
        \bottomrule
    \end{tabular}
\end{table}

\subsection{可视化结果}
插入图片示例
\begin{figure}[h]
    \centering
    \includegraphics[width=0.8\textwidth]{example-image}
    \caption{训练损失曲线}
    \label{fig:loss}
\end{figure}

\section{结果分析与讨论}
\subsection{结果分析}
对实验结果进行详细分析:
\begin{itemize}
    \item 分析不同模型的性能差异及其原因
    \item 讨论超参数对模型性能的影响
    \item 识别模型的优势和局限性
\end{itemize}

\subsection{问题与改进}
讨论实验过程中遇到的问题和可能的改进方向...

\section{实验总结}
\subsection{实验收获}
总结通过本次实验获得的知识和技能...

\subsection{心得体会}
分享实验过程中的思考和感悟...

\begin{thebibliography}{9}
\bibitem{ref1} 作者姓名. 文献标题[J]. 期刊名称, 年份, 卷号(期号): 页码.
\bibitem{ref2} 作者姓名. 书名[M]. 出版地: 出版社, 年份.
\end{thebibliography}

\end{document}